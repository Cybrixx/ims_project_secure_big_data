%**********************************************************************
% Various LaTeX packages
%**********************************************************************


% programming (branching) logic
\usepackage{ifthen}

% you might need some mathematical expressions:
\usepackage{amsmath}

% with package babel we allow to use language english and german
\usepackage[english,german]{babel}


% permits to set space between lines
\usepackage{setspace}  	

% ensure proper appearance of all fonts in pdf:
\usepackage[T1]{fontenc}

% lmodern after T1 fontenc (_may_ be required)
% lmodern = Latin Modern fonts
%\usepackage{lmodern}  	

%\usepackage{times} -- obsolete; use:
% Times as default text font, maths support
\usepackage{mathptmx}  	
% provides bold font (required for syntax highlighting in listings)
\usepackage{courier}  	

% enables table cells to span multiple rows
\usepackage{multirow}  
% paragraphs: no indentation at beginning, but spacing between  
\usepackage{parskip}  	

% for figures
\usepackage[pdftex]{graphicx}
% implicit file name extensions for embedded figures 
% so we do not need to specify the extension on inclusion
\DeclareGraphicsExtensions{.pdf,.jpg,.png}

% for flexible tables (e.g. auto resizing to page width) 
%     e.g.: { | l | l | X | }
\usepackage{tabularx}

%**********************************************************************
% Including non-standard packages
%**********************************************************************

% abbreviations are written in full length on first time usage
% e.g. with \acro{UX}{User Experience} 
%    \ac{UX} first time: User Experience (UX)
%    \ac{UX} later:      UX 
\usepackage{acronym}

% http://en.wikibooks.org/wiki/LaTeX/Colors
\usepackage[usenames,dvipsnames,table]{xcolor}
% we define a few custom colors  
\definecolor{gray20}{gray}{0.8}
\definecolor{gray5}{gray}{0.95}
\definecolor{olivegreen30}{RGB}{155,187,89}  

\usepackage{alltt}

% for code snippets, embedded as "listings"
\usepackage{listings}
% we set a few defaults below with \lstset:

  
% the page layout (geometry), 
% as defined by FH guidelines (2.3 Formale Gestaltung):
\usepackage[top=3cm, 
            bottom=3cm, 
            left=3.5cm, 
            right=3cm]
           {geometry}

% superscript st first,
%             nd second
%             rd third
%             th fourth,...
\usepackage[super]{nth}     % 1st, 2nd, 3rd,...

%\usepackage{paralist}   	% inline lists
%\usepackage{mdwlist}
\usepackage{enumitem}

% e.g. for "floating" listings (no fixed anchor in text)
\usepackage{float}     
\floatstyle{plain}
\restylefloat{figure}
% e.g. to show two (floating) images side-by-side
\usepackage{subfig}

% add copyright information to figures
\usepackage{copyrightbox} 



% symbols such as \texttimes and \texteuro
\usepackage{textcomp}  
% math. symbols from the American Mathematical Society  
%\usepackage{amssymb}  	


% chapter heading styles
% see 3.2 The Chapter Lenny at
%         https://ctan.org/pkg/fncychap
\usepackage[Lenny]{fncychap}  

% for \enquote, \textquote, \blockquote...
\usepackage[autostyle]{csquotes}       

% how to create simple helper commands (LaTeX "macros"):
% e.g. red-coloured text when using ...
%      \TODO{Do not forget to add further references}
\newcommand{\TODO}[1]
{
{\textcolor{red}{[TODO: #1]}}
}


% Create/style more complex new commands:
% e.g. \chapquote{Phone home!}{by E.T.}
% BEGIN: chapquote
\newcommand{\chapquote}[2]  
{%
\begin{quote}
\emph{%
``#1''%
}%
\begin{flushright}
{\scriptsize \sffamily [#2]}%
\end{flushright}
\end{quote}
}
% END: chapquote



% Biblatex = bibliography for LaTeX
% ---------------------------------------------------------------------
% context sensitive quotation; recommended for usage with Biblatex
\usepackage{csquotes}  	
% Note: \date, \origdate, \eventdate, and \urldate 
%       require "yyyy-mm-dd" format,
%       so "dd" or "mm-dd" may be omitted
\usepackage[backend=biber,
            urldate=long,  	        % [Feb. 8, 2027]
                                    %                  default: short 
                                    %                  e.g. 08/15/2010
            style=authoryear-icomp, % Harvard citation style
            backref,                % if you like (cit. on p. 2)
            % sorting=nty, 	        % this is default: sort by name,
                                    %                  title, year
            % sortlocale=de_DE,     % set according to your needs
            natbib=true,  	        % use "natbib"-compatible citation
                                    %                  commands
                                    % do _not_ use package natbib!
            maxbibnames=1000,  	    % show all authors in the 
                                    %                  bibliography
            % ibidpage=true,        % this is default for ibid / 
                                    %                  ebd ("ebenda")                        
            giveninits=true,         % E. B. White instead of Elwyn Brooks White
            uniquename=init,
]{biblatex}

% using & instead of "and"
\renewcommand*\finalnamedelim{\addspace\&\space}

% We enforce strict Harvard style: 
% The URL date default is "(Visited on ...);" => so:
%     BibTeX entries such as:
%      url = {http://...},
%      urldate = {2015-05},
%      urldate = {2015},
%      urldate = {2015-05-17},
%     shall be printed as
%      Available from: <http://...> [May 17, 2015]
\DeclareFieldFormat{urldate}{\mkbibbrackets{#1}}
\ifthenelse{\equal{\yourLanguage}{german}}{
  % for language = german: 
  % [online ] https://dl.acm.com/paper3.pdf [17. Mai 2015]
  \DeclareFieldFormat{url}{[online]\space \url{#1}}
}{ % else: 
   % default language = english
   % Available from: https://dl.acm.com/paper3.pdf [May 17, 2015]
  \DeclareFieldFormat{url}{Available\space from\addcolon\space \url{#1}}
}




% ---------------------------------------------------------------------
% !!
% hyperref should be last package loaded
% !! 
\usepackage[  	              
    pdftitle={\title},
    pdfsubject={\thisDocumentIsA},
    pdfauthor={\yourName},
    pdftex,  	              % driver
    breaklinks,  	          % permits line breaks for long links
    bookmarks,  	          % create Adobe bookmarks
    bookmarksnumbered,        % ... and include section numbers
    linktocpage,  	          % the page number (not the text) is link on TOC
    colorlinks,  	          % 
    linkcolor=black,          % normal internal links;
    anchorcolor=black,        % don't make scientific papers too colourful
    citecolor=black,
    urlcolor=blue,  	      % blue is quite common for urls
    pdfstartview={Fit},       % "Fit" fits the page to the window
    pdfpagemode=UseOutlines,  % open bookmarks in Acrobat
    plainpages=false,         % avoids duplicate page number problem
    pdfpagelabels,
  ]{hyperref}

% we break our rules from above and specify another package AFTER hyperref
% allow to specify multiple refs at once, using \Cref{ }
% e.g. \Cref{lst:hello,lst:closure}
\usepackage{cleveref}


%**********************************************************************
% A hack to allow the urls to break on some more characters
%**********************************************************************

% Uncomment, if you want to break URLs at . @ / ! _ | % ; > .... 
%\def\UrlBigBreaks{\do\.\do\@\do\\\do\/\do\!\do\_\do\|\do\%\do\;\do\>\do\]%
% \do\)\do\,\do\?\do\'\do\+\do\=\do\#\do\:\do@url@hyp}

% Uncomment, if you want to break URLs at 123456789 
%\def\UrlBreaks{\do\1\do\2\do\3\do\4\do\5\do\6\do\7\do\8\do\9}

% Uncomment, if you want to break URLS at abcde....xyz and ABCDE....XYZ  
%\def\UrlBreaks{\do\a\do\b\do\c\do\d\do\e\do\f\do\g\do\h\do\i\do\j\do\k\do\l%
%               \do\m\do\n\do\o\do\p\do\q\do\r\do\s\do\t\do\u\do\v\do\w\do\x%
%               \do\y\do\z%
%               \do\A\do\B\do\C\do\D\do\E\do\F\do\G\do\H\do\I\do\J\do\K\do\L%
%               \do\M\do\N\do\O\do\P\do\Q\do\R\do\S\do\T\do\U\do\V\do\W\do\X%
%               \do\Y\do\Z%
%}

% To break URLs in the bibliography
% "bib url lc penalty" will add a breakpoint after all lowercase letters. 
\setcounter{biburllcpenalty}{7000} 
% "bib url uc penalty" will add a breakpoint after all lowercase letters.
\setcounter{biburlucpenalty}{8000} 



%**********************************************************************
% Layout adjustments
%**********************************************************************

% page layout (header/footer and page numbers) 
\pagestyle{headings} % options:
         % headings
         % fancy
         % empty

% Optimise headers / footers
%   when printing the first page of a (numbered) chapter 
%   suppress the page number (on center bottom)
\newcommand{\chapterstart}{\thispagestyle{empty}}
%   when printing the first page of the bibliography
%   suppress the page number (on center bottom)
\AtBeginBibliography{\thispagestyle{empty}}

         
% settings for structure and numbering 
%  we allow three levels within text:  1.2.1 
\setcounter{secnumdepth}{3}
%  but we show two levels in TOC: 1.2
\setcounter{tocdepth}{1}



% command "\chapterend" to close a chapter 
% (flush, i.e. print remaining figures and tables)
\newcommand{\chapterend}
           {\newpage{
              \pagestyle{empty}
               \cleardoublepage
             }
           }

% footnotes: no indent, hanging
\usepackage[hang,flushmargin]{footmisc}

% new environment for smaller paragraphs
% e.g. \begin{spar}A paragraph with some indentation.\end{spar}
\newenvironment{spar}
{\begingroup \leftskip 0.7cm \rightskip\leftskip}
{\par \endgroup}
% ^^^ must be set here (or use empty line)


%**********************************************************************
% LaTeX macros and commands to style the ISBN and color code listings
%**********************************************************************



% Bibliography: distinguish between cited and non-cited entries
%               so it is possible to show non-cited entries as 
%               Further Readings
\DeclareBibliographyCategory{cited}
\AtEveryCitekey{\addtocategory{cited}{\thefield{entrykey}}}

% Bibliography: we create links for given ISBN
\DeclareFieldFormat{isbn}{\isbn{#1}}
\newcommand{\isbn}[1]
{%
{%
\ifpdf
{\small ISBN}
\href{https://isbnsearch.org/isbn/#1}{#1}%
\else
{\small ISBN}
#1%
\fi
}%
}


% You might define support for further programming languages
% when using listings
\usepackage{color}
\definecolor{lightgray}{rgb}{.9,.9,.9}
\definecolor{darkgray}{rgb}{.4,.4,.4}
\definecolor{purple}{rgb}{0.65, 0.12, 0.82}
\lstdefinelanguage{JavaScript}{
  keywords={break, case, catch, continue, debugger, default, delete, do, else, false, finally, for, function, if, in, instanceof, new, null, return, switch, this, throw, true, try, typeof, var, void, while, with},
  morecomment=[l]{//},
  morecomment=[s]{/*}{*/},
  morestring=[b]',
  morestring=[b]",
  ndkeywords={class, export, boolean, throw, implements, import, this},
  keywordstyle=\color{blue}\bfseries,
  ndkeywordstyle=\color{darkgray}\bfseries,
  identifierstyle=\color{black},
  commentstyle=\color{purple}\ttfamily,
  stringstyle=\color{red}\ttfamily,
  sensitive=true
}

\lstset{numbers=left, 
        basicstyle=\footnotesize\ttfamily,  
        showstringspaces=false,
        % numbers=none           % disable line numbering
        captionpos=b,            % caption at bottom
        breaklines=false, 
        numbersep=5pt,           % space for numbers
        caption={Listing subtitles could and should contain whole sentences
                 describing the important aspect of the listing.}, 
        float=tbhp,             % float listing to top/bottom/here/page
        language=JavaScript,     % Python Ruby SQL ksh erlang ...
        frame=single,
        breaklines=true,         % break long source code lines, add arrow
        postbreak=\mbox{\textcolor{red}{$\hookrightarrow$}\space},
        basewidth={0.55em}, 
}



%**********************************************************************
% Special hyphenation rules
%**********************************************************************

\hyphenation{JOANNEUM}  	% extend to your needs

%**********************************************************************
% Different settings for ITM / SWD / IRM / IMS
%**********************************************************************


% ITM = Internettechnik
% ------------------------
\ifthenelse{\equal{\study}{ITM}}{
  \def \theStudyProgramme {Internettechnik}
  \def \isBachelorThesis {}
}
%\fi

% SWD = Software Design
% ------------------------
\ifthenelse{\equal{\study}{SWD}}{
  \def \theStudyProgramme {Software Design}
  \def \isBachelorThesis {}
}

% MSD = Mobile Software Development
% ------------------------
\ifthenelse{\equal{\study}{MSD}}{
  \def \theStudyProgramme {Mobile Software Development}
  \def \isBachelorThesis {}
}


% IRM = IT-Recht & Management
% -------------------------------
\ifthenelse{\equal{\study}{IRM}}{
  \def \theStudyProgramme {IT-Recht \& Management}
  \def \isMasterThesis {}
}

% IMS = IT & Mobile Security
% ------------------------------
\ifthenelse{\equal{\study}{IMS}}{
  \def \theStudyProgramme {IT \& Mobile Security}
  \def \isMasterThesis {}
}


